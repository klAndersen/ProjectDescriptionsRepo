%\section{section5}
\label{sec:section5}

\subsection[Global Software Development]{\getprefix{imt4112}}
\label{sec:glob_soft_dev}
% space for readability
\begin{itemize} 
	\item \getprefix{level}: \getprefix{master}
	\item \getprefix{type}: \getprefix{group}, \getprefix{ass}
	\item \getprefix{college}: \getprefix{ntnu}
	\item \getprefix{ide}: \getprefix{pycharm}, \getprefix{al}
	\item \getprefix{progLang}: \getprefix{py}
\end{itemize} 
% space for readability
% TODO: Re-view and re-write this
Group project course where the goal was to teach us how it was to work in a global team not having all members around, and also working on expanding an existing system. 
The task my group was given was to develop a multi-choice questionnaire plugin for the Learning Management System (LMS) Open edx. 
\vspace{0.5em}\newline
Aside from creating the pages and functionality for creating, editing and displaying a questionnaire, we were given three tasks 
(the third was never started, and thus is excluded from this summary). 
In this questionnaire, students should be able to select their confidence level in regards to their submitted answers. 
Depending on the confidence level selected (0 = low, 1 = normal, 2 = high), their total score could be increased (or decreased if wrong answer was selected).
\vspace{0.5em}\newline
The second task was peer-review, where the students could select whether or not they thought the rest of the class found this to be a difficult question. 
Bonus points were then given to the students who selected the one that got the most selections.
\vspace{0.5em}\newline
Aside from handling administrative tasks (I was the Scrum Master in this project), the parts I worked on was the creation of answer alternatives, 
the answers submitted by the students and and grade for the submitted questionnaires (with confidence level). 
In the second task, I worked on the part related to difficulty level and updating of the students score.
During this project, I used Arch Linux as OS, PyCharm as IDE and Python as programming language.

\subsection[Advanced Project Work]{\getprefix{imt5251}}
\label{sec:adv_proj_work}
% space for readability
\begin{itemize} 
	\item \getprefix{level}: \getprefix{master}
	\item \getprefix{college}: \getprefix{ntnu}
	\item \getprefix{ide}: \getprefix{pycharm}, \getprefix{al}
	\item \getprefix{progLang}: \getprefix{py}
	\item \getprefix{ghrepo} \url{https://github.com/klAndersen/IMT5251_AdvProjWork}
\end{itemize} 
% space for readability
% TODO: Re-view and re-write this
Preliminary work for my Master thesis, which was mostly focused on creating a simplistic prototype. 
The topic for my Master thesis was to develop a Chat Agent that could answer students programming questions,  and also help them learn to be better at asking good questions. 
The Chat Agent was to be implemented as a module (an XBlock) in the Learning Management System (LMS) Open edx. 
Python was used, as this is the programming language which the XBlocks are written in, using  PyCharm as editor. 
\vspace{0.5em}\newline
During this development, a simple chat interface was created which could connect to StackOverflow.com. 
One could ask questions, but it only returned the first question/answer it found on Stack Overflow. 
The prototype was also presented to two students that gave feedback and suggestions for features. 
The plan for further development was to implement AI in my Master thesis using a hybrid algorithm of Hidden-Markov-Model and Bayes Net. 

\subsection{Master thesis: Predicting coding question quality}
\label{sec:master_thesis}
% space for readability
\begin{itemize} 
	\item \getprefix{level}: \getprefix{master}
	\item \getprefix{college}: \getprefix{ntnu}
	\item \getprefix{course}: \getprefix{imt4904}
	\item \getprefix{ide}: \getprefix{pycharm}, \getprefix{al}
	\item \getprefix{progLang}: \getprefix{py2}, \getprefix{py3} (submitted version)
	\item \getprefix{ghrepo} \url{https://github.com/klAndersen/IMT4904_MasterThesis_Code}
\end{itemize} 
% space for readability
% TODO: Re-view and re-write this
During the presentation of the thesis topic, I was told that my topic was not just too large for a Master thesis, but even too large for a Ph.d. thesis. 
It therefore needed to be reduced drastically, which ended up with the topic being an attempt to predict question quality at Stack Overflow by looking at existing questions. 
\vspace{0.5em}\newline
In my thesis, I used Arch Linux as an operative system since it was easier to develop for Python. 
I used MySQL as a database, which contained all the data from Stack Overflow (downloaded from StackExchange Archive). 
Data was extracted from the database into Python using a library called pandas, which also had the ability to export the data to a CSV file. 
\vspace{0.5em}\newline
The questions were originally in HTML format, so they were therefore cleansed using Beautiful Soup 4 (bs4). 
To get the vocabulary over all the words used in all the selected questions, CountVectorizer from the machine learning library sklearn was used. 
Features were extracted and replaced with feature detectors, and these were then used to create a model for predicting the question quality. 
