%\section{section8}
\label{sec:section8}

\subsection{Alumni Community}
\label{sec:alumni}
% space for readability
\begin{itemize} 
	\item \getprefix{level}: \getprefix{bach}
	\item \getprefix{type}: \getprefix{ass}
	\item \getprefix{college}: \getprefix{buc}
	\item \getprefix{course}: \getprefix{courseCode}
	\item \getprefix{ide}: \getprefix{ideName}
	\item \getprefix{progLang}: \getprefix{langName}
	\item \getprefix{ghrepo} \url{https://github.com/klAndersen/Bachelor-development-projects/tree/master/Web%20Development/Inf%20268%20Utviklingsprosjekt}
\end{itemize} 

developed from scratch, without use of external frameworks or libraries. The goal was to develop
an alumni community web page for the college. A requirement was that the delivered solution
should include HTML, CSS3, JavaScript and PHP. For storing data, MySQL was selected as
database. To have a local server running, Apache XAMPP was used. It was also a requirement
that the pages should look as identical as possible regardless of the web browser that was used (I
compared Chrome, Internet Explorer, Opera and Firefox).
The course was split into two semesters, and the development project into five parts:
1. Web page design/layout and policy
2. Access Control
3. Functionality and database model
4. Posts and user administration
5. Completion
Users were to be divided into user groups; registered, moderator and administrator. Adminstrators
was able to appoint new moderators and edit users, with the same rights as moderators.
All registered users should be able to...
• ...register, login, update their profile and be able to change/request a new password (if
forgotten).
• ... upload a profile picture, or select one of the existing ones that had previously been
uploaded to the alumni page.
• ... join or leave networks created within the alumni community. Users should be able to
message each other, and also see a notification when new messages has arrived.
• ... see a list of the other registered users.
Moderators should be able to...
• ...edit users profile (e.g. if inappropriate data was entered on their profile).
• ...notify users they had breached the guidelines or block out/quarantine users (users was
not to be deleted).
• ...create/edit/delete existing topic fields (e.g. events, area of expertise, etc).
In my project I mostly used PHP and HTML. Every page was created as a .php, as this allowed me
to update the page layout in just one file, instead of having to update all the pages. The layout was
then included at the location where it was to be used. It also simplified the use of function calls and
checks to see if a user was logged in, and had the proper rights to view a given page. Database
access was also handled via php, and JavaScript was not largely used (mostly to verify user input).
The following is a short summary of what the delivered version included. For new users, they had a
registration form, that gave feedback both with JavaScript and PHP if an error occurred. If the user
continued, even though one or more fields contained errors, the given field(s) would have a text
marked in red next to it, explaining what was wrong. Users also got their own profile, which allowed
them to update their information, alongside changing their password (users that weren't logged in
could request a new password if they forgot it). Moderators also had the ability to edit user profiles,
e.g. if any added something in-appropriate.
Users could also send messages to each other, and to ensure that the users would see if they had
received a new message, the link to the messages in the navigation bar was updated with a
number, indicating how many new, unread messages they had received. Users could message
each other via their profile, or by searching for them from the member list (which listed all, or thosematching the search critera; name or e-mail). Users who broke the page policy could be put in
quarantine. This rejected the users attempt to login, and giving them a message that they were
quarantined. An information message could also be set, that would explain to the given user why
they were quarantined.
Moderators could register campus, area of expertise and networks (e.g. degree, courses, etc).
Moderators could also create events. All of the aforementioned could be updated or deleted. For
events, only those active were displayed (but administrators and moderators could see all on the
creation page). Events that were out of date was not showed, and this was also true for future
events (based on start date for the given event).

\subsection{Flea Market}
\label{sec:flea_market}
% space for readability
\begin{itemize} 
	\item \getprefix{level}: \getprefix{bach}
	\item \getprefix{type}: \getprefix{ass}
	\item \getprefix{college}: \getprefix{buc}
	\item \getprefix{course}: \getprefix{courseCode}
	\item \getprefix{ide}: \getprefix{ideName}
	\item \getprefix{progLang}: \getprefix{langName}
	\item \getprefix{ghrepo} \url{https://github.com/klAndersen/Bachelor-development-projects/tree/master/Web%20Development/Inf%20329%20XML%20-%20Oblig.%20oppg}
\end{itemize} 

\subsection[Applied Computer Science Project]{\getprefix{imt4003}}
\label{sec:appl_comp_sci_proj}
% space for readability
\begin{itemize} 
	\item \getprefix{level}: \getprefix{master}
	\item \getprefix{type}: \getprefix{group}, \getprefix{ass}
	\item \getprefix{college}: \getprefix{ntnu}
	\item \getprefix{ide}: \getprefix{ideName}
	\item \getprefix{progLang}: \getprefix{langName}
\end{itemize} 

\subsection[Integration project]{\getprefix{imt4004}}
\label{sec:key}
% space for readability
\begin{itemize} 
	\item \getprefix{level}: \getprefix{master}
	\item \getprefix{type}: \getprefix{ass}
	\item \getprefix{college}: \getprefix{ntnu}
	\item \getprefix{ide}: \getprefix{ideName}
	\item \getprefix{progLang}: \getprefix{langName}
	\item \getprefix{ghrepo} \url{https://github.com/klAndersen/IMT4004-Integration-Project}
\end{itemize} 