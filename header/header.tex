\documentclass[a4paper]{article}
%% Languages
\usepackage[utf8]{inputenc}
\usepackage[english]{babel}
%% splits long URLs; 
%% see http://tex.stackexchange.com/questions/3033/forcing-linebreaks-in-url
\PassOptionsToPackage{hyphens}{url}
\usepackage{hyperref}
%% word indexing
\usepackage{makeidx}
%% quotations
\usepackage{csquotes}% Recommended
%% Graphics and colors
\usepackage[svgnames]{xcolor}
\usepackage{listings}
\usepackage{graphicx}
\usepackage{tikz}
\usepackage{color}
%% Math Packages
\usepackage{amsmath}
\usepackage{amsfonts}
%% to be able to write multi-line comments, display text "as-is", etc.
\usepackage{verbatim}
%%packages for displaying multiple figures side by side
\usepackage{caption}
\usepackage{subcaption}
%% use url's
\usepackage{url}
%% used to be able to call the name of the chapter instead of the number
\usepackage{nameref}
%% to be able to change bulletpoint look in lists
\usepackage{enumitem}
%% use tikz library for figures, nodes, images, etc.
\usetikzlibrary{shapes.geometric,arrows}
%% valid extensions for import in figures, etc.
\DeclareGraphicsExtensions{.pdf,.png,.jpg,.jpeg,.bmp}
%% decrease size of listing font
\lstset{%
basicstyle=\ttfamily\footnotesize\bfseries,
columns=fullflexible}
%% colours the links in the pdf
\hypersetup{
  colorlinks = true, %Colours links instead of ugly boxes
  urlcolor = blue, %Colour for external hyperlinks
  linkcolor = black, %Colour of internal links
  citecolor = red %Colour of citations
}

%% Prefixes to avoid repetitive typing (e.g. school name, programming languages, etc)
\newcommand\setprefix[2]{\expandafter\def\csname#1\endcsname{#2}}
\newcommand\getprefix[1]{\csname#1\endcsname}

%% for creating indexes
\makeindex
