%\section{section3}
\label{sec:section3}

\subsection[C++ course]{\getprefix{ite1546}}
\label{sec:cpp_programming}
% space for readability
\begin{itemize} 
	\item \getprefix{level}: \getprefix{bach}
	\item \getprefix{type}: \getprefix{ass}s
	\item \getprefix{college}: \getprefix{nuc}
	\item \getprefix{ghrepo} \url{https://github.com/klAndersen/Bachelor-development-projects/tree/master/C%2B%2B/ITE1546%20Programmering%20i%20C%2B%2B}
\end{itemize} 
% space for readability
This was an introductory course, targeted at the Electrical engineering students. 
Therefore it was only commandline and no GUI, using using native C++.
There were three mandatory assignments, and one graded.

\paragraph{\getprefix{ass} 1} ~\\
The first assignment was focused on understanding the basics, and creating a simple commandline menu:
\begin{enumerate}
	\item Calculate the volume of a ball/sphere (based on entered radius).
	\item Solve a quadratic equation (based on entered a, b and c).
	\item Calculate distance between two geographical points (using Haversine Formula).
	\item Calculate and print repayment plan for annuity loan.
	The user enters the home loans, interest rates (in \%) and the number of years for repayment. 
	\item Convert a number (integer) to text (e.g. 31 = thirty).
\end{enumerate}

\paragraph{\getprefix{ass} 2} ~\\
The second was based on creating a simple cinema ticket system, which only had to register one performance (the user was the ticket seller).
The cinema contained 15 rows, and 30 seats, where vacant seats were marked with '*' and sold seats were marked with a '\#'.
The cinema operated with two prices: full price (e.g. £9) and reduced prices (e.g. £6).
The menu with its requirements are listed below: 
\begin{enumerate}
	\item New performance: 
	Sales data was stored in textfile, so that it could be re-used on startup. 
	This file was only reset when selecting this option. 
	This option set all seats to vacant, and asked for the name of the new performance.
	\item Buy Tickets: 
	Allows user to select desired row and column for seating, in addition to registering which price the ticket(s) were sold for.
	After confirming sale, the selected seats were marked as sold.
	\item Show vacant seats:
	Prints a view of the current performance, showing which seats are  vacant.
	\item Display ticket sales for this performance:	
	Show the number of tickets sold per day, income per row and total income on current performance.
	\item End program
\end{enumerate}

\paragraph{\getprefix{ass} 3} ~\\
The third was based on Bitmap pixel colour manipulation, where one should be able to alter the colour of all pixels, or grey-scaling them all.
The assignment also required that they were based on the following two prototypes:
\begin{lstlisting}
// component: 0 = R, 1 = G, 2 = B
void changeColor(vector<Pixel> &pixelTable, int component, int colour) 

void greyScale(vector<Pixel> &pixelTable)
\end{lstlisting}

\subsubsection{Graded assignment: Recipe Program}
\label{sec:recipe_program}
A chef wants a program to keep track of nutritional information related to meals that he composes. 
By using this program, he should be able to compose dishes and get calculated corresponding nutrition information as energy in kcal (and/or Joule) and the content of proteins,
fat, carbohydrates, vitamins, etc.
This data was exctracted from an Excel file (converted into CSV-format), which contained nutritional values for more than 1300 different food types (per 100 grams of edible product). 
\vspace{0.5em}\newline
The chef envisions that he chooses x grams of ingredient1\footnote{
	The given ingredient is selected based on its food number.
}, y grams of ingredient2, z grams of ingredient3, etc. to compose a dish and then (or continuously) get printed total nutrient for the new dish.  
The recipe contains quantity (in grams) for each food and overall nutrition information (amount of protein). 
Each new finished dish should be written to a recipe file (a separate file per dish).
\vspace{0.5em}\newline
The program should also print out different reports:
\begin{itemize}
	\item All registered dishes/recipes listed on the screen.
	\item Dishes with more than x (determined by user) g protein, fat or carbohydrates per 100g done right.
	\item Dishes with more than x (determined by user) calories per 100g finished right.
	\item Dishes with lots of vitamin D (compared to the daily recommended intake).
	\item Dishes with much vitamin A (in relation to the daily recommended intake).
	\item Dishes with much cholesterol (compared to the daily recommended intake).
\end{itemize}
% space for readability
\getprefix{ghrepo} \url{https://github.com/klAndersen/Bachelor-development-projects/tree/master/C%2B%2B/ITE1546%20Programmering%20i%20C%2B%2B/Karaktergivende}

\subsection{Machine Learning library}
\label{sec:ml_library}
% space for readability
\begin{itemize} 
	\item \getprefix{level}: \getprefix{master}
	\item \getprefix{type}: Developed for re-usability in assignments
	\item \getprefix{college}: \getprefix{guc}
	\item \getprefix{course}: \getprefix{imt4612}
	\item \getprefix{ghrepo} \url{https://github.com/klAndersen/Machine-Learning/tree/master/MachineLearning}
\end{itemize} 
% space for readability
% TODO: Re-view and re-write this
This was written for use in the course "\getprefix{imt4612}". 
It is not a complete project, as it is mostly built on the algorithms taught in the course, and those that were asked for during the mandatory assignments. 
Aside from algorithms, it contains a Conversion class (string to numeric, native to Managed C++, vica versa, etc), a Matrix class, a class for Statistics and Exceptions.
% TODO: Add short summary of what it includes (e.g. which algorithms that were "finished")

\subsection[Computational Forensics]{\getprefix{imt4641}}
\label{sec:comp_forensics}
% space for readability
\begin{itemize} 
	\item \getprefix{level}: \getprefix{master}
	\item \getprefix{type}: \getprefix{ass}
	\item \getprefix{college}: \getprefix{guc}
	\item \getprefix{ide}: \getprefix{vs}, \getprefix{sqlite}, Microsoft HTML Help Workshop
	\item \getprefix{ghrepo} \url{https://github.com/klAndersen/IMT4641-Computational-Forensics}
\end{itemize} 
% space for readability
% TODO: Re-read this, and potentially add something about the code logic
This course was development only, where we were tasked with coming up with our own idea to develop a tool or system that was relevant to the courses we had.
My project focused on creating a re-active tool (used after an incident has occurred) that analysed Android databases (SQLite). 
A requirement was that the database(s) was already extracted from the given device(s) that were under investigation.
\vspace{0.5em}\newline
After selecting a location containing database(s) to investigate, the user has the option to add filters to their search. 
Since the path could contain more then one database, they were given the option to select all or just a few. 
Search categories and types could also be defined (e.g. a category could be "Internet", and a type could be ".html").
In addition, option to select a timeline (start/end date) were also included. 
This way, if there was a specific day or week being investigated, this could reduce the amount of results that needed analyzing.
\vspace{0.5em}\newline
The result screen contains a TreeView which lists the folder(s), database(s) and the database(s) table(s). 
The folder is the parent node, where the database is a child of the folder and the table(s) are the child(ren) of the database.
If the investigator was looking for something specific, the investigator could search through the TreeView.
Matches were then highlighted in yellow.
\vspace{0.5em}\newline
The content of a given databases table was listed in a DataGridView, which listed the columns and rows.
Databases that were empty, only contained one row in the table 'android metadata', or did not match the search criteria(s) was excluded from the result. 
If the investigators found any databases they wanted to take a closer look at, they could just rightclick the database in the TreeView, and select "Open folder".


\begin{comment}

\subsection{mlAssignment1}
\label{sec:mlAssignment1}
% space for readability
\begin{itemize} 
	\item \getprefix{level}: \getprefix{master}
	\item \getprefix{type}: \getprefix{ass}
	\item \getprefix{college}: \getprefix{guc}
	\item \getprefix{course}: \getprefix{imt4612}
	\item \getprefix{ghrepo} \url{xxx}
\end{itemize} 

\subsection{mlAssignment2}
\label{sec:mlAssignment2}
% space for readability
\begin{itemize} 
	\item \getprefix{level}: \getprefix{master}
	\item \getprefix{type}: \getprefix{ass}
	\item \getprefix{college}: \getprefix{guc}
	\item \getprefix{course}: \getprefix{imt4612}
	\item \getprefix{ghrepo} \url{xxx}
\end{itemize} 

\subsection{mlAssignment3}
\label{sec:mlAssignment3}
% space for readability
\begin{itemize} 
	\item \getprefix{level}: \getprefix{master}
	\item \getprefix{type}: \getprefix{ass}
	\item \getprefix{college}: \getprefix{guc}
	\item \getprefix{course}: \getprefix{imt4612}
	\item \getprefix{ghrepo} \url{xxx}
\end{itemize} 

\end{comment}
