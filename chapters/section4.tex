%\section{section4}
\label{sec:section4}

\subsection{Animal Registration}
\label{sec:animal_registration}
% space for readability
\begin{itemize} 
	\item \getprefix{level}: \getprefix{bach}
	\item \getprefix{type}: \getprefix{ass}
	\item \getprefix{college}: \getprefix{buc}
	\item \getprefix{course}: INF240 and \getprefix{inf244} 
	\item \getprefix{ide}: \getprefix{eclp}, \getprefix{mysqlw}
	\item \getprefix{progLang}: \getprefix{java}
	\item \getprefix{ghrepo} \url{https://github.com/klAndersen/Bachelor-development-projects/tree/master/Java/Oblig2_Inf240_KnutLucasAndersen}
	\item \getprefix{ghrepo} \url{https://github.com/klAndersen/Bachelor-development-projects/tree/master/Java/Oblig_Inf244_KnutLucasAndersen}
\end{itemize} 
% space for readability
In the two Java courses in my Bachelor, we had the same assignment, but with different requirements. 
The main focus was to register two different animals (hares and lynx). 
Common for both was the registration of their gender (male/female), length (Double), weight (Double), time of capture (day, month, year) and location (String). 
The ID was to be incremented, and start with the initial for the given animal (e.g. L1, L2, \ldots, H1, H2, \ldots). 
For the hares, colour (String) and type (Char) was to be registered. 
For the lynx, the length of their ears (Double).
\vspace{0.5em}\newline
If an animal was later re-captured, their data should be updated instead of re-registered. 
The program had the possibility to search for animals based on ID, show amounts of re-captures (based on the entered year), 
show amount of different hare captures (based on type), and an unsorted report of all first-time captures.
\vspace{0.5em}\newline
The first course was introductory level, so input/output was shown in command line, and data was stored in a text file. 
In the second course, the user was presented with a GUI that we had to code ourselves (not use a Designer), and the data was stored in a MySQL Database 
(using JDBC as database driver). 
Singleton was used to ensure that there was only one object maintaing the database connection. 
The main menu used JFrame, and the child windows used JDialog.

\subsection[Android course]{\getprefix{ite1621}}
\label{sec:android_course}
% space for readability
\begin{itemize} 
	\item \getprefix{level}: \getprefix{bach}
	\item \getprefix{type}: \getprefix{ass}
	\item \getprefix{college}: \getprefix{nuc}
	\item \getprefix{ide}: \getprefix{eclp}, \getprefix{mysqlw}
	\item \getprefix{progLang}: \getprefix{java}, \getprefix{jserv}, \getprefix{android} 2.3 - 4.1, \getprefix{xml}, \getprefix{sqlite}
	\item \getprefix{ghrepo} \url{https://github.com/klAndersen/Bachelor-development-projects/tree/master/Android/ITE1621%20Applikasjoner%20for%20mobil%20og%20web}
\end{itemize} 
% space for readability
This was an introductory course to developing applications for Android. 
It consisted of four mandatory assignments and a graded assignment. 
\vspace{0.5em}\newline
The first assignment was creating a conversion app for different measure systems (distance, volume, mass, temperature and time).

% TODO: Taken from the original task description
Within each of the main categories are also included examples of different devices that program to calculate from/to. 
The user begins by selecting the main category, then select the device to be counted from, unit to be counted and the numerical value to be converted. 
The program make the conversion and displays the result on the screen.
To use Fragments and the application will use a different layout on small and large screens. 
On the big screen should be left of the screen, a list of categories that you can choose from. 
Right shall as corresponding subcategory to/from ETC. see you later. The program will also be multilingual ie minimum Norwegian and English.
Examples: 
If the user selects area as a general category, one can convert from/to devices such as square millimeters, squarecentimeters, squaredecimeters, Square feet, 
squareinches, square yards, Acres, etc. 
Selected category Mass shall be converted from/to micrograms, milligrams, centigram, ounces, pounds etc. 
Selected temperature it should be possible to convert from/to Celsius, Fahrenheit or Kelvin.
\vspace{0.5em}\newline
% TODO: Taken from the original task description
It will develop an application (A) which can be used to manage information about books/book titles. 
The program will save the info. on an application associated text file. 
The application shall be (at least) the following activities (Activity classes):
* An activity that offers an interface that displays a list of all registered titles. 
The user can choose a book title and this should then be returned from this activity.
* An activity that allows you to register a new book title, this is stored on the file.
* An activity that allows you to delete a selected book title.
* Start-up activity should have a set of buttons or a menu that provides access to the other activities.

Activities in application A will then be used by another application (B). 
From this application, the user should make up the list which the activity of A, with titles and could choose and return selected title back to B. 
Selected title displayed in the application B. In addition, the user could choose to add a new title, using the activity of application A.

Every time it added a new book title sends a broadcast of application A. 
This should be captured by the application B and displayed by means of a Toast.
The application B must therefore contain a broadcast receiver, receiving broadcasts from application A.
Feel free to use Fragments where appropriate.
\vspace{0.5em}\newline
% TODO: Taken from the original task description
The task is to develop two Android applications, one that logs and stores information. about various "events" and one that serves as an interface to the log.
---
Application 1: Event Logs
This application is by means of a  BroadcastReceiver to capture and store info from system events such. receiving SMS, phone calls, GPS turned on or off, WiFi turned on or off, etc. 
All events are stored in a SQLite database.
Each event should be associated with a category (eg. Telephony, network, positioning and the like, you determine categories and which events will be in the category).
Each event will be stored with date, time, category, text, details (special parameters / values associated with the event, eg. Who SMS comes from).
When starting the application, a simple "Setting" activity appears which allows the user to decide which events (of a range of at least 5 different events from at least 
3 different categories) to be logged.
The application should also work so that when a kringkasing occurs, the current broadcast receiver start and store relevant information in the database. 
The application should not be started for it to work. 
The database must be "exposed" by means. a Content Provider when it will be available for Application 2 - LoggerUI.
---
Application 2: LoggerUI
This will use info from database in app1 and display this in a simple Activity. 
Here user can select the category and display all events from this category (eg. from last night or the like). 
Here, use a ContentResolver.
\vspace{0.5em}\newline
% TODO: Re-view and re-write this
The fourth assignment was creating a weather notification app. 
The weather app downloaded XML data from a weather page in set intervals based on settings set by user. 
The data was then parsed and displayed to user, and the user could set a temperature limit (min/max). 
If the temperature was out of the set range, a notification should be shown to the user.
---
% TODO: Taken from the original task description
It will develop an Android application that will monitor the temperature at different measurement stations used by southwest. 
A weather station has a unique number and name (eg. 84700, Narvik airport). 
Of applications shall include a service that periodically downloads weather data (observations) from yr.no and possible. 
Notify (using. Notification) if the temperature is outside set limits.
In addition to service shall also contain an application Activity showing a list of measurement stations that the user wants to monitor. 
Beside the station name will last interrogated temperature displayed. 
The user must via a menu option to add (and possibly remove) monitor station. 
A weather station has a name, a measureNo and a URL.
Different measuring stations can have the same URL.

After the service is started will periodically download temperature data. 
URL of giving forecasts for Narvik, pick out the temperature of the current station by means of XML parsing.
The activity shall have the following menu options:
Adding and deleting measuring stations. 
A weather station recorded with three values: place names, målernr and URL. 
Can be stored on file or in a SharedPreference.

Settings
Here one should be able to set the interval to download weather data (eg. every hour, every two hours, every three hours, etc.).
In addition, it should be possible to set upper and lower limit temperatures. 
If downloaded spot temperature is above/below these values, the service should give a notification.
From the activity should also be possible to start and stop the service (should be visible "actions" in ActionBar'en). 
The service will continue to run in the background even if the event ends. 
Make sure the service starts when the device starts.

\subsubsection{TracknHide}
\label{sec:tracknhide}
This project was part of the course grade, and consisted of developing an application that could show the different users location and route on a map. 
The project had two parts, an Android application and a Java web server.
\vspace{0.5em}\newline
The Android application used Google Cloud Messaging (GCM) to send and receive data, and Google Maps (API v1) to present a map. 
By using MapOverlay, users could also show the adress when clicking a location on the map, and switch between Street – and SateliteView. 
On the map, the route and current position of logged in users with shared position was shown. 
Users could also choose to store their own or others route, which could later then be re-drawn on the map. 
The position data was retrieved by using the devices GPS, but the amount of position data sent was set by the user (amount of time passed or distance moved).
\vspace{0.5em}\newline
The Java Web server was created using HTML, Java Servlets and JSP, running on Apache TomCat v7.0. 
It stored the user related information in a MySQL Database, and kept track over all the currently logged in users. 
The web pages was mainly targeted at an administrator, to give them the ability to view connected users, and disconnect if needed. 
The server was also the connection point for the users to share/receive position data and notifications via GCM.
\getprefix{ghrepo} \url{https://github.com/klAndersen/Bachelor-development-projects/tree/master/Android/ITE1621%20Applikasjoner%20for%20mobil%20og%20web/Karaktergivende%20oppgave} 

\subsection{TrackMyTeacher}
\label{sec:trackmyteacher}
% space for readability
\begin{itemize} 
	\item \getprefix{level}: \getprefix{master}
	\item \getprefix{type}: \getprefix{ass}
	\item \getprefix{college}: \getprefix{ntnu}
	\item \getprefix{course}: \getprefix{imt5401}
	\item \getprefix{ide}: \getprefix{as}, \getprefix{mysqlw}
	\item \getprefix{progLang}: \getprefix{java}, \getprefix{android}, \getprefix{xml}, \getprefix{php}, JSON
	\item \getprefix{ghrepo} \url{https://github.com/klAndersen/IMT5401-Mobile-Research}
\end{itemize} 
% space for readability
In this course I tried to develop an indoor tracking prototype app for Android. 
The purpose was to be able to locate teachers and professors on campus, by measuring the WiFi-signals at their current location 
(if they had enabled/allowed tracking of their current location).
\vspace{0.5em}\newline
The server side of the application was developed using PHP (because the college server only ran PHP), which mainly handled data transfer between the Android app and the MySQL database.
When data from the database was needed, the server retrieved the data, added it to an array as JSON and printed it as HTML. 
This HTML was then converted to a JSONArray on the mobile device (as I stated in my report, it would probably have been much easier if one could have used Java Servlets, 
since one could overwrite the HTTP GET/POST methods).
\vspace{0.5em}\newline
To get test data, I went to different buildings, rooms and floors spread out on campus (20 rooms, signals measured in all 4 corners plus the middle of the room). 
What I found when looking at the gathered data was that the measurement for the WiFi-signal was set too low (value=5). 
Therefore this prototype was not finished, as I did not have the time to re-measure all the rooms to get more accurate signal data.

\subsection{BeregnSnitt}
\label{sec:beregnsnitt}
% space for readability
\begin{itemize} 
	\item \getprefix{type}: \getprefix{priv}
	\item \getprefix{ide}: \getprefix{eclp}
	\item \getprefix{progLang}: \getprefix{java}
	\item \getprefix{ghrepo} \url{https://github.com/klAndersen/Bachelor-development-projects/tree/master/Java/beregnSnitt}
\end{itemize} 
% space for readability
When nearing the end of my Bachelor degree, I wanted to know what my average grade was, to see what colleges and universities I could apply to in regards to taking a Master degree. 
This lead to the development of "BeregnSnitt". 
BeregnSnitt calculates the average score based on the grades and total points the user has.
\vspace{0.5em}\newline
The user is presented with a GUI (Jframe and Jdialog), where s/he can enter the amount of A to E's achieved, the course points and the total points achieved (e.g. Bachelor = 180). 
I also added the possibility to add additional course grades, in case some courses varied (e.g. if the standard was 10, but one course had 3 points, another 5, etc.).

\subsection{Auto-reply for Android}
\label{sec:auto_reply_android}
% space for readability
\begin{itemize} 
	\item \getprefix{type}: \getprefix{priv}
	\item \getprefix{ide}: \getprefix{eclp}, \getprefix{as}\footnote{
		The development started with using Eclipse, but was later changed to using Android Studio as IDE.
	}
	\item \getprefix{progLang}: \getprefix{java}, \getprefix{android}, \getprefix{xml}
\end{itemize} 
% space for readability
% TODO: Re-view and re-write this
This was a program I started on after learning Android. 
The reason I started working on it was because one often gets text messages or calls when you cannot reply or answer. 
Sometimes, people continue to text or call, and I therefore started working on what I call an Auto-reply app. 
The auto-reply is sent out as a text message (SM)S, which could either be random or a set topic.
\vspace{0.5em}\newline
The auto-reply could be sent to only to those marked for it, or apply to all in the phones contact list. 
The auto-reply was restricted by a delay, adjustable through the settings. 
When a call or SMS came in, if the person calling/texting was on the Auto-reply list, a timer would start. 
When the timer ran out, a check would occur to see if the SMS/lost call was unread. If so, a SMS would be sent out. 
\vspace{0.5em}\newline
The SMS that were sent from the Auto-reply are random, but that was mostly because I created it for myself. 
I planned to later add some Machine Learning logic to it, so that it could reply somewhat intelligently to the SMS, instead of a random text.
\vspace{0.5em}\newline
