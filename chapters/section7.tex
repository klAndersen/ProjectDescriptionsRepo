%\section{section7}
\label{sec:section7}

Suggestions for sections/content:
\begin{itemize} 
	\item \getprefix{inf340}
	\item \getprefix{ite1607} (udk/unreal)
	\item \getprefix{imt4006}
	\item \getprefix{imt4032}
	\item \getprefix{imt4072}
	\item \getprefix{imt4122}
	\item \getprefix{imt4612}: assignments
	\item anything else\ldots?
\end{itemize} 


\subsection{Teacher-Student questionnaire system}
\label{sec:inf116_teacher_system}
% space for readability
\begin{itemize} 
	\item \getprefix{level}: \getprefix{bach}
	\item \getprefix{type}: \getprefix{group}, \getprefix{ass}
	\item \getprefix{college}: \getprefix{buc}
	\item \getprefix{course}: \getprefix{inf116}
	\item \getprefix{ide}: \getprefix{vs}, \getprefix{mysqlw}
	\item \getprefix{progLang}: \getprefix{vb}
\end{itemize} 
% space for readability
% TODO: Re-view and re-write this
A two-part group project, where the first part was the teacher system and the second the system for the students.
I was part of the group working with the teacher system (we were 3 people working on this), where the system should be able to create new questions for a given course, 
and set parameters for the evaluation. 
The teacher would need to get a statistical overview of students answers, and also be able to export/store these to a file. 
The tools used were Visual Studio IDE, Visual Basic as programming language (course requirement), and MySQL as database. 
The GUI was Windows Forms.
\vspace{0.5em}\newline
% TODO: Taken from the original task description
It should be made an application for evaluation of course. 
The application shall be made with the database in MySQL and application in Visual Basic 2008 (gladly Express version). 
It is emphasized that the application is neat, clean and robust in use with good online user assistance. 
It is also emphasized that it is used recognized principles of good code. "Report Designer" should not be used. 
Programming community's auto with "DataSet" (available in full version of VB) will not be used - it must therefore explicitly created the necessary objects in the program.
---
The system should have two main parts:
1. For teachers:
a. Management Section where teachers can create questionnaires
b. report stage where teachers get out statistics about students' answers
2. For students: Besvarelsesdel where students can fill out the individual forms part of the evaluation
---
Information about students and courses are taken electronically from student system, which, inter alia, states which students taking a given course 
(a "course" is a subject / course that taught a given semester, for example. INF110 V07). 
The model for this looks like this:

---
For each evaluation made a number of questions for each question is given two to five answer options that students can choose from. 
The teacher determines all the details of the evaluation, including which course it applies, what the evaluation will be called, 
what questions the evaluation should contain and what response options to be granted. 
The sub-model for this part looks like this:
---
The attributes evalDatoUt and evalDatoInn specify the time when the evaluation will be available for students. 
The teacher should be able to change the questions and answers, but not after the first student has completed the evaluation.
---
There may be multiple evaluations for each course, and they may overlap in time:
---
Once the student starts the system, he will sign in and be presented with the reviews that are relevant to him. 
An evaluation is appropriate only if:
1 student taking the course 
2. the student does not already have evaluated the course 
3. current date is within the period from evalDatoUt even evalDatoInn
---
When the student has completed the evaluation, he will receive a receipt on the screen. The sub-model for answers, looks like this:
---
In the report section teacher should be able to learn how answers were distributed for each question. 
It should also be possible to know who has completed evaluation (it is mandatory to carry out certain evaluations). 
There should be an opportunity to "export" the counted data for a specific evaluation to a flat file, which can then be imported into Excel. 
The teacher should also be able to delete all student responses for an evaluation (when it is finished).
---
The database is attached as SQL scripts, with some data. 
Note in particular the creation of a user "Knut" with password "Tunk". 
The user uses you when the application will be linked to the database. 
Feel free to add more data, but you can not modify the database structure, name or the like. 
You can not save queries (views) in the database for use in the program - but you can create views with the create statements in your program.

\subsection[System Developing Project]{\getprefix{inf165}}
\label{sec:inf165_sys_dev}
% space for readability
\begin{itemize} 
	\item \getprefix{level}: \getprefix{bach}
	\item \getprefix{type}: \getprefix{group}, \getprefix{ass}
	\item \getprefix{college}: \getprefix{buc}
	\item \getprefix{ide}: \getprefix{macc}, \getprefix{mysqlw}
\end{itemize} 
% space for readability
% TODO: Re-view and re-write this
This was a group project in a system development course. 
The task was to develop a system for hotel management, where the user should be able to check-in/out guests, see available rooms, room expenses and retrieve economical reports.
The system was created by using Microsoft Access and MySQL as database. Microsoft Visio was used to create DataFlow Diagrams.
\vspace{0.5em}\newline
% TODO: Taken from the original task description
Storenuten Høgfjellshotel is a mountain hotel with excellent location in the Norwegian mountains. 
For without their own family-friendly resorts have mountain hotel responsibility ramp of 130 km ski slopes in the surrounding areas. 
It is a short distance to municipal newly established state of the art resorts. 
Høyfjellshotell has since the 50s evolved from a small family business to now being a major employer in the county with pepole. 
The hotel is newly renovated and has all the rights. 
---
The hotel has 130 modern double rooms and 35 family rooms with 4 beds with the possibility of an extra bed. 
When it comes to stay in double or family so include award full board (breakfast, lunch and dinner). 
The price is per date 1020 kr for an adult per day, and it is half price for children under 16 years. 
Beverages for lunch and dinner are not included. 
The hotel follows common practice to "write expenditure Room" (the guest acknowledges the talon inflicted room number), 
ie the settlement of drinks to food, something good from the bar, mini bar, etc. may be by settlement / out.
---
After market entry of the ski resort has mountain hotel experienced strong growth in its business.
They now have an urgent need for a new order and stay system for hotel operations. 
On the ground it is important that the new system handles the work of managing orders, check in and stay ( "expenses are recognized in the room during the stay") and Settlement / out.
---
On the basis of the information at any time in the system will be possible to take out information of a financial nature.

\subsection[Flash Programming]{\getprefix{inf330}}
\label{sec:flash_prog}
% space for readability
\begin{itemize} 
\item \getprefix{level}: \getprefix{bach}
\item \getprefix{type}: \getprefix{group}, \getprefix{ass}
\item \getprefix{college}: \getprefix{buc}
\item \getprefix{ide}: \getprefix{afcs}, \getprefix{fldev}
\item \getprefix{progLang}: \getprefix{flash}, \getprefix{as3}
\end{itemize} 

\subsubsection{Space Invaders}
\label{sec:space_invaders}
% TODO: Re-view and re-write this
A scroll-game like space invaders, where the player controls a space-ship. 
This was developed in Flash, using ActionScript 3 as scripting language. 
The topic for the development was chosen by the group. 
A short scenario context is that the player is a farmer kidnapped by aliens, and have to save the Earth (and his harvest).
\vspace{0.5em}\newline
I handled the design and development of the enemies, game settings and playing sound/music. 
The game contained up to six levels, where the player could select which to play (but level 2-6 had to be opened first by completing previous level. 
The player got points for destroying enemies (asteroids and aliens). 
After two minutes of play, the boss appeared. 
The game also contained a "Top 10" list displaying the scores for the players.

\subsubsection{Tower of Knowlegde}
\label{sec:tower_of_knowledge}
% TODO: Re-view and re-write this
Tower defence game meets quiz where the goal is to attack only the wrong answers. 
The topic was selected by the group. 
The player can put out none or more towers, and the player had to let the "correct" answer only pass.
\vspace{0.5em}\newline
Answers was shot by clicking the answer, presuming it was within the given towers range. 
Points were given for attacking enemies, and reduced for each answer passing the goal. 
The player was not told if the answer was correct until the game had ended, and when the game was over, the player was added to the Top 10 list presuming their score was high enough. 
My responsibilities in this project was the development of the main menu, reading (questions, answers, Top 10) and writing (Top 10) to file, sound/music.
\vspace{0.5em}\newline
The game was developed using Flash and ActionScript. The IDE was Adobe Flash CS 4 and FlashDevelop (actionscript). 
In addition, I also created a small program for the creation of questions and answers (since the variables was read from file using URLVariables), which was written in Visual Studio, C\#.
\vspace{0.5em}\newline
This is how the file content looked like:
\textit{qst=question1?,question2?\&alt=alt1,alt2\&ans=ans1,ans2}

\subsection[3D Modelling]{\getprefix{ite1606}}
\label{sec:ite1606_3d_modelling}
% space for readability
\begin{itemize} 
	\item \getprefix{level}: \getprefix{bach}
	\item \getprefix{type}: \getprefix{ass}
	\item \getprefix{college}: \getprefix{nuc}
	\item \getprefix{ide}: \getprefix{3dsm}
	\item \getprefix{ghrepo} \url{https://github.com/klAndersen/Bachelor-development-projects/tree/master/3D%20Modeling/ITE1606%203Dmodellering}
\end{itemize} 
% space for readability
% TODO: Re-view and re-write this
This course was based on learning how to create 3D-models and how to work with a 3D-modeling tool, namely 3DS MAX. 
This was a purely introductory course, so I cannot claim to have any expertise after this course, other than a general understanding.
\vspace{0.5em}\newline
In the first two assignments, we were asked to create a sketch of a car, and create textures based on taking images of clothing, carpets, etc. 
The next four assignments were based on creating various models (a Lego block with our name on it using the texture we created), a car, 
a stadium for the car to drive on and a character representation of ourselves).
\vspace{0.5em}\newline
The last two assignments covered Skeletal movement (making the character able to move, and give it a short animation) and lighting 
(we had to simulate "evening sun" without using plug-ins or "daylight system").
\vspace{0.5em}\newline
I also had a course in Game Design, where we used UDK and Unreal to model a game world (the code was written using Unreal Script).
\vspace{0.5em}\newline
% TODO: Taken from the original task description
task1 \\
The task is to model a Duplokloss (with 8 buds on top). 
It should look something like the following figures. 
See also the web for pictures of bricks to get the appropriate details. 
Like the real blocks is important. In my models depicted below, there are some shortcomings and you can do better. 
On the one hand should your name be in attendance for "Lego" that I have put on my.
---
Add an appropriate texture for the group as block appears authentic. 
Add in simple colors / textures that resemble blocks of reality. 
Texturing can take much time, but it is not the most important in the exercise, so consider how much time you spend on this.
\vspace{0.5em}\newline
% TODO: Taken from the original task description
Based on what is covered in Module 1 and 2 shall be used to model your vehicle as you outlined in Module 0 Exercise 1 Sketch. 
You must model with a view to future use so vehicle propulsion (wheels, belts, bone, etc.) can be animated at a later date. 
The same applies to the vehicle's opportunities for course change / control.
---
That all submissions should have roughly the same size ratio between the different vehicles and for subsequent scaling in such XNA / C\#, 
go into the menu "Customize-> Units Setup" and select the "System Units Setup." 
Here you select the "1Unit = Centimeter" (press ok). 
Secondly, select the "Display Unit Scale" should be Metric and set to Centimeters. 
"Lighting Units" may be International.
---
Then draw a box with dimensions (xyz) 200,200,100 located in coordinates (XYZ) 0,0,0. 
This should be maximum (bounding box) for your model. The box is only to give the impression of how big your model like that about to be. 
The box allows you to hide (hide) and remove when needed.
\vspace{0.5em}\newline
% TODO: Taken from the original task description
Based on what is covered in Module 1 and 2 and 3 shall be used to model "race cart / vehicle trail" with trbiune and facilities to your vehicle that you modeled in 
"Module 2 Exercise 4 Vehicles".
---
You must model considering scaling the vehicle. 
If you use the same Unit setup as outlined in rehearsal "Module 2 Exercise 4 Vehicles", but changes the Display Units Scale to "Meters", 
then draw a box located to the (xyz) 0,0,0 with size 150m, 150m, 30m has a region that restricts "bounding box for the environment.
---
The box is only to give the impression of how big your model like that about to be. 
The box allows you to hide (hide) and remove when needed. 
Insert your vehicle onto the "path" after you have modeled the finished
\vspace{0.5em}\newline
% TODO: Taken from the original task description
Based on what is undergone in module 1..4 you should model a lavpolygon character (Avatar) by yourself. 
You assumes a tutor who came in an earlier version of 3dsmax, but we will turn the focus away from modulate a "Soldier" and using instead pictures of yourself 
to modeled one LowPoly model yourself after the existing this tutor. 
The tutor can be found by downloading tutorfiler from version 9sp2 of 3DSMax located at this link: \url{http://fagweb.hin.no/axs/kilder/3dsmax_t.chm}. 
REMEMBER, that after you have downloaded it you must right-click the file, select properties for the file and choose "unblock" or you will not see its contents. 
Once that's done, open the file and navigate to "Modelling" -> "Modeling a Low-Poly Character". 
Get help from someone to take pictures of yourself from the front and from the side. 
Make sure you do not go crazy when it comes to polygons, you should in turn right shape and animate it as a helper for the vehicle in the path you created in previous exercises. 
Use the same scale as the one you used to model vehicle. 
The avatar must fit the size of the vehicle.
\vspace{0.5em}\newline
% TODO: Taken from the original task description
Based on what is undergone in module 1..5 should you rig your avatar from previous practice with a skeleton. 
Please note that there is no talk of a biped here. 
You must use the "bones" and then the modifier "skin" after which you need to adjust "envelope" of the individual members / bones 
so that the skeleton motion moving the right parts of the figure. 
Add arm and fotrigger (as shown at the end of chapter 23 in the book) and Inverse Kinematics (IK) that you can use in animation.
\vspace{0.5em}\newline
% TODO: Taken from the original task description
Based on what is undergone in module 1..6 light the composite path, vehicle and character in a mood corresponding "evening sun" (see the book chapter 25). 
Rendering must function without error messages. This is an exercise in using bright spots. 
You should not take advantage of "Daylight System"
