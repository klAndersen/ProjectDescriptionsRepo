%\section{section3}
\label{sec:section3}

\subsection[C++ course]{\getprefix{ite1546}}
\label{sec:cpp_programming}
% space for readability
\begin{itemize} 
	\item \getprefix{level}: \getprefix{bach}
	\item \getprefix{type}: \getprefix{ass}s
	\item \getprefix{college}: \getprefix{nuc}
	\item \getprefix{ghrepo} \url{https://github.com/klAndersen/Bachelor-development-projects/tree/master/C%2B%2B/ITE1546%20Programmering%20i%20C%2B%2B}
\end{itemize} 
% space for readability
% TODO: Re-view and re-write this
This was a pure introductory course, as it was targeted at the students studying electronics. 
Therefore there was no GUI, only command line in Visual Studio, using native C++.
There were three mandatory assignments, and one graded. 
\vspace{0.5em}\newline
The first assignment was mainly focusing on creating a menu-program that could execute some equations, and convert numbers to their textual value. 
The second was based on creating a cinema ticket system, which should only register one performance, sales data and vacant sales, etc, which was stored in a text file. 
If a new performance was registered, the old was deleted. 
The third was based on Bitmap pixel colour manipulation, where one should be able to alter the colour of all pixels or grey-scaling them all.

\subsubsection{Graded assignment: recipeProgram} %TODO: Find name of program
\label{sec:recipe_program}
\getprefix{ghrepo} \url{https://github.com/klAndersen/Bachelor-development-projects/tree/master/C%2B%2B/ITE1546%20Programmering%20i%20C%2B%2B/Karaktergivende}
% NEWLINE
% TODO: Re-view and re-write this
The graded assignment consisted of creating a system that could store recipes, calculate a recipes nutrition data, list the data for a given nutrition, 
list recipes containing more than X nutritions, etc. 
The nutrition data was extracted from an Excel-file provided by the teacher.

\subsection{Machine Learning library}
\label{sec:ml_library}
% space for readability
\begin{itemize} 
	\item \getprefix{level}: \getprefix{master}
	\item \getprefix{type}: Developed for re-usability in assignments
	\item \getprefix{college}: \getprefix{guc}
	\item \getprefix{course}: \getprefix{imt4612}
	\item \getprefix{ghrepo} \url{https://github.com/klAndersen/Machine-Learning/tree/master/MachineLearning}
\end{itemize} 
% space for readability
% TODO: Re-view and re-write this
This was written in Visual Studio 2013 using Managed C++, to use in the Machine Learning course I had. 
It is not a complete project, as it is mostly built on the algorithms taught in the course, and those that were asked for during the mandatory assignments. 
Aside from algorithms, it contains a Conversion class (string to numeric, naitive to Managed C++, vica versa, etc), a Matrix class, a class for statistics and exceptions.

\subsection[Computational Forensics]{\getprefix{imt4641}}
\label{sec:comp_forensics}
% space for readability
\begin{itemize} 
	\item \getprefix{level}: \getprefix{master}
	\item \getprefix{type}: \getprefix{ass}
	\item \getprefix{college}: \getprefix{guc}
	\item \getprefix{ide}: \getprefix{vs}, \getprefix{sqlite}
	\item \getprefix{ghrepo} \url{https://github.com/klAndersen/IMT4641-Computational-Forensics}
\end{itemize} 
% space for readability
% TODO: Re-view and re-write this
This course was development only, where we were tasked with coming up with our own idea to develop a tool or system that was relevant to the track and courses we had.
My project focused on creating a re-active tool (used after an incident has occurred) that analysed Android databases (SQLite). 
It was developed using Visual Studio C++, using Windows Forms as GUI.
\vspace{0.5em}\newline
I re-used parts of the graphical design that I used in my Bachelor thesis, since some qualities were similar. 
The tool lists the folders, SQLite database(s) and tables in a TreeView.
In the Treeview, the folder is the parent, and the database(s) are the children. 
The table(s) in turn, are the child(ren) of the selected database. 
When a table was selected, the data was listed in a DatagridView. 
One could also search for a given entry in the listed treeview, where matches were highlighted in yellow.

\begin{comment}

\subsection{mlAssignment1}
\label{sec:mlAssignment1}
% space for readability
\begin{itemize} 
	\item \getprefix{level}: \getprefix{master}
	\item \getprefix{type}: \getprefix{ass}
	\item \getprefix{college}: \getprefix{guc}
	\item \getprefix{course}: \getprefix{imt4612}
	\item \getprefix{ghrepo} \url{xxx}
\end{itemize} 

\subsection{mlAssignment2}
\label{sec:mlAssignment2}
% space for readability
\begin{itemize} 
	\item \getprefix{level}: \getprefix{master}
	\item \getprefix{type}: \getprefix{ass}
	\item \getprefix{college}: \getprefix{guc}
	\item \getprefix{course}: \getprefix{imt4612}
	\item \getprefix{ghrepo} \url{xxx}
\end{itemize} 

\subsection{mlAssignment3}
\label{sec:mlAssignment3}
% space for readability
\begin{itemize} 
	\item \getprefix{level}: \getprefix{master}
	\item \getprefix{type}: \getprefix{ass}
	\item \getprefix{college}: \getprefix{guc}
	\item \getprefix{course}: \getprefix{imt4612}
	\item \getprefix{ghrepo} \url{xxx}
\end{itemize} 

\end{comment}
