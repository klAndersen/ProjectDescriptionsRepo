%\section{section7}
\label{sec:section7}

Suggestions for sections/content:
\begin{itemize} 
	\item \getprefix{inf340}
	\item \getprefix{ite1607} (udk/unreal)
	\item \getprefix{imt4006}
	\item \getprefix{imt4032}
	\item \getprefix{imt4072}
	\item \getprefix{imt4122}
	\item \getprefix{imt4612}: assignments
	\item anything else\ldots?
\end{itemize} 


\subsection{Teacher-Student questionnaire system}
\label{sec:inf116_teacher_system}
% space for readability
\begin{itemize} 
	\item \getprefix{level}: \getprefix{bach}
	\item \getprefix{type}: \getprefix{group}, \getprefix{ass}
	\item \getprefix{college}: \getprefix{buc}
	\item \getprefix{course}: \getprefix{inf116}
	\item \getprefix{ide}: \getprefix{vs}, \getprefix{mysqlw}
	\item \getprefix{progLang}: \getprefix{vb}
\end{itemize} 
% space for readability
% TODO: Re-view and re-write this
A two-part group project, where the first part was the teacher system and the second the system for the students.
I was part of the group working with the teacher system (we were 3 people working on this), where the system should be able to create new questions for a given course, 
and set parameters for the evaluation. 
The teacher would need to get a statistical overview of students answers, and also be able to export/store these to a file. 
The tools used were Visual Studio IDE, Visual Basic as programming language (course requirement), and MySQL as database. 
The GUI was Windows Forms.
\vspace{0.5em}\newline
% TODO: Taken from the original task description
It should be made an application for evaluation of course. 
The application shall be made with the database in MySQL and application in Visual Basic 2008 (gladly Express version). 
It is emphasized that the application is neat, clean and robust in use with good online user assistance. 
It is also emphasized that it is used recognized principles of good code. "Report Designer" should not be used. 
Programming community's auto with "DataSet" (available in full version of VB) will not be used - it must therefore explicitly created the necessary objects in the program.
---
The system should have two main parts:
1. For teachers:
a. Management Section where teachers can create questionnaires
b. report stage where teachers get out statistics about students' answers
2. For students: Besvarelsesdel where students can fill out the individual forms part of the evaluation
---
Information about students and courses are taken electronically from student system, which, inter alia, states which students taking a given course 
(a "course" is a subject / course that taught a given semester, for example. INF110 V07). 
The model for this looks like this:

---
For each evaluation made a number of questions for each question is given two to five answer options that students can choose from. 
The teacher determines all the details of the evaluation, including which course it applies, what the evaluation will be called, 
what questions the evaluation should contain and what response options to be granted. 
The sub-model for this part looks like this:
---
The attributes evalDatoUt and evalDatoInn specify the time when the evaluation will be available for students. 
The teacher should be able to change the questions and answers, but not after the first student has completed the evaluation.
---
There may be multiple evaluations for each course, and they may overlap in time:
---
Once the student starts the system, he will sign in and be presented with the reviews that are relevant to him. 
An evaluation is appropriate only if:
1 student taking the course 
2. the student does not already have evaluated the course 
3. current date is within the period from evalDatoUt even evalDatoInn
---
When the student has completed the evaluation, he will receive a receipt on the screen. The sub-model for answers, looks like this:
---
In the report section teacher should be able to learn how answers were distributed for each question. 
It should also be possible to know who has completed evaluation (it is mandatory to carry out certain evaluations). 
There should be an opportunity to "export" the counted data for a specific evaluation to a flat file, which can then be imported into Excel. 
The teacher should also be able to delete all student responses for an evaluation (when it is finished).
---
The database is attached as SQL scripts, with some data. 
Note in particular the creation of a user "Knut" with password "Tunk". 
The user uses you when the application will be linked to the database. 
Feel free to add more data, but you can not modify the database structure, name or the like. 
You can not save queries (views) in the database for use in the program - but you can create views with the create statements in your program.

\subsection[Hotel management system]{\getprefix{inf165}: Hotel management system}
\label{sec:inf165_sys_dev}
% space for readability
\begin{itemize} 
	\item \getprefix{level}: \getprefix{bach}
	\item \getprefix{type}: \getprefix{group}, \getprefix{ass}
	\item \getprefix{college}: \getprefix{buc}
	\item \getprefix{ide}: \getprefix{macc}, \getprefix{mysqlw}, Microsoft Visio (DataFlow Diagrams)
\end{itemize} 
% space for readability
Storenuten Høgfjellshotel is a mountain hotel with an excellent location in the Norwegian mountains. 
Aside from their own family-friendly resorts, the hotel has responsibility for maintaining 130 km ski slopes in the surrounding areas.
It is a short distance to the municipals newly established state of the art resorts.
Høyfjellshotell has since the 50s evolved from a small family business to now being a major employer in the county with people. 
The hotel is newly renovated and has all the rights. 
\vspace{0.5em}\newline
The hotel has 130 modern double rooms and 35 family rooms with 4 beds, with the possibility of an extra bed. 
When it comes to stay in double or family, the price includes full board (breakfast, lunch and dinner; beverages for lunch and dinner are not included). 
The price is per date 1020 kr for an adult per day, and it is half price for children under 16 years. 
The hotel writes expenses to the room (the guest fills out receipt with room number; expenses can be drinks, mini-bar, etc.), which is then settled at check-out.
\vspace{0.5em}\newline
After market entry of the ski resort, the mountain hotel has experienced strong growth in its business.
They now have an urgent need for a new system to handle their hotel operations. 
In the first round, it is important that the new system handles the work of managing orders, check-in and stay ("expenses are recognized in the room during the stay") and check-out.
On the basis of the information at any time in the system will be possible to take out information of a financial nature.

\subsection[Flash Programming]{\getprefix{inf330}}
\label{sec:flash_prog}
% space for readability
\begin{itemize} 
\item \getprefix{level}: \getprefix{bach}
\item \getprefix{type}: \getprefix{group}, \getprefix{ass}
\item \getprefix{college}: \getprefix{buc}
\item \getprefix{ide}: \getprefix{afcs}, \getprefix{fldev}
\item \getprefix{progLang}: \getprefix{flash}, \getprefix{as3}
\end{itemize} 

\subsubsection{Space Invaders}
\label{sec:space_invaders}
% TODO: Re-view and re-write this
A scroll-game like space invaders, where the player controls a space-ship. 
The topic for the development was chosen by the group. 
A short scenario context is that the player is a farmer kidnapped by aliens, and have to save the Earth (and his harvest).
\vspace{0.5em}\newline
I handled the design and development of the enemies, game settings and playing sound/music. 
The game contained up to six levels, where the player could select which to play (but level 2-6 had to be opened first by completing previous level. 
The player got points for destroying enemies (asteroids and aliens). 
After two minutes of play, the boss appeared. 
The game also contained a "Top 10" list displaying the scores for the players.

\subsubsection{Tower of Knowlegde}
\label{sec:tower_of_knowledge}
% TODO: Re-view and re-write this
Tower defence game meets quiz where the goal is to attack only the wrong answers. 
The topic was selected by the group. 
The player can put out none or more towers, and the player had to let the "correct" answer only pass.
\vspace{0.5em}\newline
Answers was shot by clicking the answer, presuming it was within the given towers range. 
Points were given for attacking enemies, and reduced for each answer passing the goal. 
The player was not told if the answer was correct until the game had ended, and when the game was over, the player was added to the Top 10 list presuming their score was high enough. 
My responsibilities in this project was the development of the main menu, reading (questions, answers, Top 10) and writing (Top 10) to file, sound/music.
\vspace{0.5em}\newline
In addition, I also created a small program for the creation of questions and answers (since the variables was read from file using URLVariables), which was written in \getprefix{csharp}.
\vspace{0.5em}\newline
Example of file content: \\
\textit{qst=question1?,question2?\&alt=alt1,alt2\&ans=ans1,ans2}

\subsection[3D Modelling]{\getprefix{ite1606}}
\label{sec:ite1606_3d_modelling}
% space for readability
\begin{itemize} 
	\item \getprefix{level}: \getprefix{bach}
	\item \getprefix{type}: \getprefix{ass}
	\item \getprefix{college}: \getprefix{nuc}
	\item \getprefix{ide}: \getprefix{3dsm}
	\item \getprefix{ghrepo} \url{https://github.com/klAndersen/Bachelor-development-projects/tree/master/3D%20Modeling/ITE1606%203Dmodellering}
\end{itemize} 
% space for readability
% TODO: Re-view and re-write this
This course was based on learning how to create 3D-models and how to work with a 3D-modeling tool (\getprefix{3dsm}). 
This was a purely introductory course, so I cannot claim to have any expertise after this course, other than a general understanding.

\paragraph{\getprefix{ass} 1-2} ~\\
In the first two assignments, the task was to create a sketch of a vehicle (I sketched a car), and create textures based on taking  and editing images of clothing, carpets, etc. 

\paragraph{\getprefix{ass} 3} ~\\
The task was to model a Lego block, which should look as realistic as possible. 
Instead of the text "Lego", our own name was to be added on the one side. 
It also contained texture, to resemble reality. 

\paragraph{\getprefix{ass} 4} ~\\
The task was to create a vehicle based on the sketch from \getprefix{ass} 1. 
The vehicle was to be modeled in such a way that it could be animated later (in regards to wheels, belts, bone, course change, etc.).

\paragraph{\getprefix{ass} 5} ~\\
The task was to model a "race cart/vehicle trail/stadium" for the vehicle to drive on, with tribune and facilities.
We also had to take scaling into consideration, in regards to the vehicle. 

\paragraph{\getprefix{ass} 6} ~\\
The task was to create an avatar model of ourselves, based on pictures of ourselves.
The scaling was the same as that for the vehicle in \getprefix{ass} 4, since the avatar had to "fit" the size of the vehicle.

\paragraph{\getprefix{ass} 7} ~\\
Based on the avatar created in \getprefix{ass} 6, the task was to rig it with a skeleton. 
This was done using the "bones" and then modifying the "skin", followed by adjusting the "envelope" of the individual members/bones.
This was to ensure that the skeleton motion moved the right parts of the figure.
The skeleton also had arms and foot-triggers, and had a short animation of a fall by using Inverse Kinematics (IK).

\paragraph{\getprefix{ass} 8} ~\\
This task focused on lighting, where the goal was to simulate "evening sun", without using plugins or taking advantage of the "Daylight System".
There were also camera usage, that showed a short animation of the vehicle (\getprefix{ass} 4) driving around on the stadium (\getprefix{ass} 5).
