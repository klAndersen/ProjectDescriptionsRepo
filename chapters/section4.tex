%\section{section5}
\label{sec:section5}

\subsection{Animal Registration}
\label{sec:animal_registration}
% space for readability
\begin{itemize} 
	\item \getprefix{level}: \getprefix{bach}
	\item \getprefix{type}: \getprefix{ass}
	\item \getprefix{college}: \getprefix{buc}
	\item \getprefix{course}: INF240 and \getprefix{inf244} 
	\item \getprefix{ide}: \getprefix{eclp}
	\item \getprefix{progLang}: \getprefix{java}
	\item \getprefix{ghrepo} \url{https://github.com/klAndersen/Bachelor-development-projects/tree/master/Java/Oblig2_Inf240_KnutLucasAndersen}
	\item \getprefix{ghrepo} \url{https://github.com/klAndersen/Bachelor-development-projects/tree/master/Java/Oblig_Inf244_KnutLucasAndersen}
\end{itemize} 
In the two Java courses in my Bachelor, we had the same assignment, but with different requirements. 
The main focus was to register two different animals (hares and lynx). 
Common for both was the registration of their gender (male/female), length (Double), weight (Double), time of capture (day, month, year) and location (String). 
The ID was to be incremented, and start with the initial for the given animal (e.g. L1, L2, ..., H1, H2, ...). 
For the hares, colour (String) and type (Char) was to be registered. 
For the lynx, the length of their ears (Double).
\vspace{0.5em}\newline
If an animal was later re-captured, their data should be updated instead of re-registered. 
The program had the possibility to search for animals based on ID, show amounts of re-captures (based on the entered year), 
show amount of different hare captures (based on type), and an unsorted report of all first-time captures.
\vspace{0.5em}\newline
The first course was introductory level, so input/output was shown in command line, and data was stored in a text file. 
In the second course, the user was presented with a GUI that we had to code ourselves (not use a Designer), and the data was stored in a MySQL Database 
(using JDBC as database driver). 
Singleton was used to ensure that there was only one object maintaing the database connection. 
The main menu used JFrame, and the child windows used JDialog.

\subsection[Android course]{\getprefix{ite1621}}
\label{sec:android_course}
% space for readability
\begin{itemize} 
	\item \getprefix{level}: \getprefix{bach}
	\item \getprefix{type}: \getprefix{ass}
	\item \getprefix{college}: \getprefix{nuc}
	\item \getprefix{ide}: \getprefix{eclp}
	\item \getprefix{progLang}: \getprefix{java}, \getprefix{android}, \getprefix{xml}
	\item \getprefix{ghrepo} \url{https://github.com/klAndersen/Bachelor-development-projects/tree/master/Android/ITE1621%20Applikasjoner%20for%20mobil%20og%20web}
\end{itemize} 

\subsubsection{TracknHide}
\label{sec:tracknhide}
\getprefix{ghrepo} \url{https://github.com/klAndersen/Bachelor-development-projects/tree/master/Android/ITE1621%20Applikasjoner%20for%20mobil%20og%20web/Karaktergivende%20oppgave}

\subsection{TrackMyTeacher}
\label{sec:trackmyteacher}
% space for readability
\begin{itemize} 
	\item \getprefix{level}: \getprefix{master}
	\item \getprefix{type}: \getprefix{ass}
	\item \getprefix{college}: \getprefix{ntnu}
	\item \getprefix{course}: \getprefix{imt5401}
	\item \getprefix{ide}: \getprefix{as}, \getprefix{mysqlw}
	\item \getprefix{progLang}: \getprefix{android}, \getprefix{xml}
	\item \getprefix{ghrepo} \url{https://github.com/klAndersen/IMT5401-Mobile-Research}
\end{itemize} 

\subsection{BeregnSnitt}
\label{sec:beregnsnitt}
% space for readability
\begin{itemize} 
	\item \getprefix{type}: \getprefix{priv}
	\item \getprefix{ide}: \getprefix{eclp}
	\item \getprefix{progLang}: \getprefix{java}
	\item \getprefix{ghrepo} \url{https://github.com/klAndersen/Bachelor-development-projects/tree/master/Java/beregnSnitt}
\end{itemize} 

\subsection{Auto-reply for Android}
\label{sec:auto_reply_android}
% space for readability
\begin{itemize} 
	\item \getprefix{type}: \getprefix{priv}
	\item \getprefix{ide}: \getprefix{eclp}, \getprefix{as}
	\item \getprefix{progLang}: \getprefix{java}, \getprefix{android}, \getprefix{xml}
\end{itemize} 

